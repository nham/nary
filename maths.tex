\input{preamble}

% OK, start here.
%
\begin{document}

\title{maths}

\maketitle


\section{misc.}

\begin{definition}
\label{def-fiber}

If $X$ and $Y$ are sets and $f: X \rightarrow Y$ is a function, then the \textbf{fiber} of $y \in Y$ is the pre-image of $y$ under $f$.

\end{definition}


\begin{definition}
\label{def-collection-all-functions}

If $X$ and $Y$ are sets, then the collection of all functions $f: X \rightarrow Y$ is denoted $Y^X$.

\end{definition}

\begin{definition}
\label{def-k-ary-operator}

Let $X$ be a set and $k \in \mathbb{P}$. A function $f: X^k \rightarrow X$ is called a \textbf{$k$-ary operator}. A $0$-ary (or nullary) operator is defined to be a singleton subset $\{x\}$ of $X$.

A function is an \textbf{operator} if it is a $k$-ary operator for some $k$. If $\phi$ is a $k$-ary operator then $k$ is called the $\textbf{arity}$ of $\phi$, notated $ar(\phi)$.

We will use the notation $f(x_1, \ldots, x_k)$ to refer to an arbitrary image of an operator even in the case that $f$ is nullary.

\end{definition}


\begin{definition}
\label{def-indexed-family}

An \textbf{index family} is a function $f: I \rightarrow \mathcal{S}$ where $I$ is any set and $\mathcal{S}$ is a collection of sets. $I$ is called the \textbf{index set}. We often forego notating the function explicitly and just refer to $\{S_i\}$ as a index family.

\end{definition}


\begin{definition}
\label{def-eqrel-congruent-operator}

If $A$ is a set and $f$ is a $k$-ary operator on $A$ and $\squiggle$ is an equivalence relation on $A$, then $\squiggle$ is \textbf{congruent with respect to $f$} if for all $(a_1, \ldots, a_k)$ and $(b_1, \ldots, b_k)$ in $A^k$ such that $a_i \squiggle b_i$, then $f([a_1], \ldots, [a_k]) \squiggle f([b_1], \ldots, [b_k])$.

This allows one to define a $k$-ary operator $F$ on $A\\ \squiggle$ by $F([a_1], \ldots, [a_k]) = [f(a_1, \ldots, a_k)]$

\end{definition}


\begin{definition}
\label{def-abstract-algebra}

A pair $(X, \{\sigma_i\})$ where $X$ is any set and $\{\sigma_i\}$ is an indexed family of operators on $X$.

If the collection of operators is finite, we sometimes notate the algebra $(X, \sigma_1, \ldots, \sigma_n)$.
    
\end{definition}


\begin{definition}
\label{def-similar-algebras}

Two algebras $(X, \{\sigma_i\})$ and $(Y, \{\phi_i\})$ are \textbf{similar} if their families of operators have the same index set $I$ and for all $i \in I$, $ar(\sigma_i) = ar(\phi_i)$.
\end{definition}

\label{def-subalgebra}

Given an algebra $\mathcal{A} = (X, \{\sigma_i\})$, a subset $S$ of $X$ is \textbf{closed under operators of $\mathcal{A}$} if for every operator $\sigma$ of $\mathcal{A}$, $\sigma(s_1, \ldots, s_k)$ is in $S$ whenevr $s_1, \ldots, s_k$ are, where $ar(\sigma) = k$.

A \textbf{subalgebra} of $\mathhcal{A}$ is a pair $(S, \{\sigma_i\})$ where $S \subseteq X$ and $S$ is closed under the operators $\sigma_i$ (i.e. the operators of $\mathcal{A}$).
\end{definition}

\begin{proposition}
\label{prop-intersection-subalgebras}

If $\mathcal{A} = (X, \{\sigma_i\})$ is an abstract algebra and $\mathcal{S}$ is a collection of subalgebras of $\mathcal{A}$, then $(\bigcap \mathcal{S}, \{\sigma_i\})$ is a subalgebra of $\mathcal{A}$.

\end{proposition}

\begin{proof}
TODO?

\end{proof}


\begin{definition}
\label{def-subalgebra-generated}

Given $(X, \{\sigma_i\})$ is an algebra, if $S$ is a subset of $X$, then \textbf{the subalgebra generated by $S$} is the intersection of all subalgebras containing $S$.

\end{definition}


\begin{definition}
\label{def-algebra-induced-from}

Given $(X, \{\sigma_i\})$ is an algebra, if $A$ is any set, then \textbf{the algebra induced from $A$} is the algebra $(X^A, \{\box_i\})$, where we define for $f_1, \ldots, f_k \in X^A$ and each $\box_i$, the function $\box_i(f_1, \ldots, f_k)(a) := \sigma_i(f_1(a), \ldots, f_k(a))$ for $\sigma_i$ with $ar(\sigma_i) = k$. This leads to a well-defined algebra on $X^A$ that is similar to the original algebra.

\end{definition}


\begin{definition}
\label{def-algebra-homomorphism}

Given $(X, \{\sigma_i\})$ and $(Y, \{\phi_i\})$ are similar algebras, then $f: X \rightarrow Y$ is called an \textbf{algebra homomorphism} if for all $i \in I$ and $x_1, \ldots, x_k$ in $X$, $f(\sigma_i(x_1, \ldots, x_k)) = \phi_i(f(x_1), \ldots, f(x_k))$ (where, as usual, $ar(\sigma_i) = k$).

\end{definition}


\begin{proposition}
\label{prop-image-of-homomorphism-is-subalgebra}

Given $(X, \{\sigma_i\})$ and $(Y, \{\phi_i\})$ are similar algebras and $f: X \rightarrow Y$ is an algebra homomorphism, then $(img f, \{\phi_i\})$ is a subalgebra of $(Y, \{\phi_i\})$.

\end{proposition}

\begin{proof}
TODO?

\end{proof}


\begin{lemma}
\label{lemma-homomorphisms-generator-unique}

If $\mathcal{A}$ and $\mathcal{B}$ are algebras and $X \subseteq \mathcal{A}$ generates $\mathcal{A}$, and if $h_1, h_2: \mathcal{A} \rightarrow \mathcal{B}$ are homomorphisms such that for all $x \in X$, $h_1(x) = h_2(x)$, then $h_1 = h_2$.

\end{lemma}

\begin{proof}

The set $\{ a \in A : h_1(a) = h_2(a) \}$ is non-empty and closed under the operations of $\mathcal{A}$, so it's a subalgebra of $\mathcal{A}$. It contains the generator $X$, so it must in fact be the whole algebra $\mathcal{A}$.

\end{proof}


\begin{definition}
\label{def-algebra-homomorphism-kernel}

Given $(X, \{\sigma_i\})$ and $(Y, \{\phi_i\})$ are similar algebras and $f: X \rightarrow Y$ is an algebra homomorphism, then the \textbf{kernel} of $f$ is the collection of all fibers of $y \in img f$.

\end{definition}


\begin{lemma}
\label{lemma-kernel-of-homomorphism-is-partition}

Given $(X, \{\sigma_i\})$ and $(Y, \{\phi_i\})$ are similar algebras and $f: X \rightarrow Y$ is an algebra homomorphism, and $K$ is the kernel of $f$, then $K$ is a partition of $X$. 

Due to the correspondence between partitions and equivalence relations, we can define an equivalence relation $\squiggle$ (TODO: fix this command) on $X$ by $a \squiggle b$ iff $f(a) = f(b)$.

\end{lemma}

\begin{proof}
TODO?

\end{proof}


\begin{definition}
\label{def-congruence-relation}
If $\mathcal{A} = (X, \{\sigma_i\})$ is an algebra and $\squiggle$ is an equivalence relation on $X$, then $\squiggle$ is a \textbf{congruence relation on $\mathcal{A}$} if $\squiggle$ is congruent with respect to each $\sigma_i$.

\end{definition}

\begin{proposition}
\label{prop-congrel-induces-similar-algebra}

If $\mathcal{A} = (X, \{\sigma_i\})$ is an algebra and $\squiggle$ is a congruence relation on $\mathcal{A}$, then $(A \\ \squiggle, \{\phi_i\})$ defined by $\phi_i([a_i], \ldots, [a_k]) := [\sigma_i(a_1, \ldots, a_k)]$ is an algebra similar to $\mathcal{A}$.
\end{proposition}

\begin{proof}
TODO?

\end{proof}


\begin{definition}
\label{def-quotient-algebra}

If $\mathcal{A} = (X, \{\sigma_i\})$ is an algebra and $\squiggle$ is a congruence relation on $\mathcal{A}$, then the similar algebra induced by $\squiggle$ is called the \textbf{quotient algebra modulo $\squiggle$}.

\end{definition}


\begin{proposition}
\label{prop-kernel-is-congrel}

If $K$ is the kernel of a homormorphism $f: \mathcal{A} \rightarrow \mathcal{B}$ between two similar algebras, then $K$ is a congruence relation.

\end{proposition}

\begin{proof}
TODO?

\end{proof}

\begin{definition}
\label{def-natural-numbers}
The \textbf{natural numbers} $\mathbb{N}$ is a set with $0 \in \mathbb{N}$ such that there is a function $s: \mathbb{N} \rightarrow \mathbb{N} - 0$ satisfying:

1. $s$ is injective
2. for all $X \subseteq \mathbb{N}$, if $0 \in X$ and $n \in X \implies s(n) \in X$, then $X = \mathbb{N}$
\end{definition}


\begin{theorem}
\label{theorem-recursion-natural-numbers}

If $X$ is a set, $f: X \rightarrow X$ a function and $a \in X$, then there is a unique function $h: \mathbb{N} \rightarrow X$ such that:

1. $h(0) = a$
2. $h(s(n) = f(h(n))$ for all $n \in \mathbb{N}$

\end{theorem}

\begin{proof}
TODO? Or omitted? It's not too hard, just slightly tedious.

Also, it uses the uniqueness of an algebra homomorphism taking fixed values on a generator,

\end{proof}

\begin{definition}
\label{def-addition-multiplication-naturals}

By the recursion theorem we can define functions $\bigtriangleup, \star: \mathbb{N}^2 \rightarrow \mathbb{N}$, called \textbf{addition} and \textbf{multiplication}, respectively, as follows. For all $m, n \in \mathbb{N}$:

1. $m \bigtriangleup 0 := m$ 
2. $m \bigtriangleup s(n) := s(m \bigtriangleup n)$
3. $m \star 0 := 0$
4. $m \star s(n) = (m \star n) \btu m$

\end{definition}

\begin{theorem}
\label{theorem-add-mult-monoids-distribute}

$(\mathbb{N}, \btu, 0)$ is a commutative monoid, $(\mathbb{N}, \star, 1)$ is a commutative monoid and for all $m, n, p \in \mathbb{N}$, $m \star (n \btu p) = (m \star n) \btu (m \star p)$.

\end{theorem}

\begin{proof}

    To prove $m \btu (n \btu p) = (m \btu n) \btu p$ for all $m, n, p \in \mathbb{N}$, we induct on $p$. $m \btu (n \btu 0) = m \btu n = (m \btu n) \btu 0)$. Also, if it's true for $p$, then $m \btu (n \btu s(p)) = m \btu s(n \btu p) = s(m \btu (n \btu p)) = s((m \btu n) \btu p) = (m \btu n) \btu s(p)$.

    To prove that $0$ is a unit, we must only establish that $0 \btu m$ for all $m$. Another induction is in order: $0 \btu 0 = 0$, and if $0 \btu n = n$, then $0 \btu s(n) = $s(0 \btu n) = s(n)$.
    
    We also need this lemma: for all $n$, $n \btu 1 = 1 \btu n$, where $1$ is defined as $s(0)$. $n = 0$ holds, and if we assume it holds for $n$, then $s(n) \btu 1 = s(n) \btu s(0) = s(s(n) \btu 0) = s(s(n)) = s(s(n \btu 0)) = s(n \btu 1) = s(1 \btu n) = 1 \btu s(n)$.

    At last we mount our attack on commutativity. Note that we have already proven that every $n \in \mathbb{N}$ commutes with both 0 and 1. Define
    
    $S = \{m \in \mathbb{N} : \forall n \in \mathbb{N} m \btu n = n \btu m\}$

    We want to prove that $S = \mathbb{N}$. $0$ (and $1$) is in $S$. Now suppose $m \in S$. To prove that $s(m) \in S$, we induct on $n \in \mathbb{N}$. (Double induction!) Clearly $s(m)$ commutes with $0$. If $s(m)$ commutes with $n$, then $s(m) \btu s(n) = s(s(m) \btu n) = s(n \btu s(m)) = s(n \btu (1 \btu m)) = s((n \btu 1) \btu m) = s(s(n) \btu m) = s(n) \btu s(m)$.

\end{proof}

\end{document}
