\input{preamble}

% OK, start here.
%
\begin{document}

\title{maths}

\maketitle


\section{misc.}

\begin{definition}
\label{def-collection-all-functions}

If $X$ and $Y$ are sets, then the collection of all functions $f: X \rightarrow Y$ is denoted $Y^X$.

\end{definition}

\begin{definition}
\label{def-k-ary-operator}

Let $X$ be a set and $k \in \mathbb{P}$. A function $f: X^k \rightarrow X$ is called a \textbf{$k$-ary operator}. A $0$-ary (or nullary) operator is defined to be a singleton subset $\{x\}$ of $X$.

A function is an \textbf{operator} if it is a $k$-ary operator for some $k$. If $\phi$ is a $k$-ary operator then $k$ is called the $\textbf{arity}$ of $\phi$, notated $ar(\phi)$.

\end{definition}


\begin{definition}
\label{def-indexed-family}

An \textbf{index family} is a function $f: I \rightarrow \mathcal{S}$ where $I$ is any set and $\mathcal{S}$ is a collection of sets. $I$ is called the \textbf{index set}. We often forego notating the function explicitly and just refer to $\{S_i\}$ as a index family.

\end{definition}


\begin{definition}
\label{def-abstract-algebra}

A pair $(X, \{\sigma_i\})$ where $X$ is any set and $\{\sigma_i\}$ is an indexed family of operators on $X$.

If the collection of operators is finite, we sometimes notate the algebra $(X, \sigma_1, \ldots, \sigma_n)$.
    
\end{definition}


\begin{definition}
\label{def-similar-algebras}

Two algebras $(X, \{\sigma_i\})$ and $(Y, \{\phi_i\})$ are \textbf{similar} if their families of operators have the same index set $I$ and for all $i \in I$, $ar(\sigma_i) = ar(\phi_i)$.
\end{definition}

\label{def-subalgebra}

Given an algebra $\mathcal{A} = (X, \{\sigma_i\})$, a subset $S$ of $X$ is \textbf{closed under operators of $\mathcal{A}$} if for every operator $\sigma$ of $\mathcal{A}$, $\sigma(s_1, \ldots, s_k)$ is in $S$ whenevr $s_1, \ldots, s_k$ are, where $ar(\sigma) = k$.

A \textbf{subalgebra} of $\mathhcal{A}$ is a pair $(S, \{\sigma_i\})$ where $S \subseteq X$ and $S$ is closed under the operators $\sigma_i$ (i.e. the operators of $\mathcal{A}$).
\end{definition}

\begin{lemma}
\label{lemma-intersection-subalgebras}

If $\mathcal{A} = (X, \{\sigma_i\})$ is an abstract algebra and $\mathcal{S}$ is a collection of subalgebras of $\mathcal{A}$, then $(\bigcap \mathcal{S}, \{\sigma_i\})$ is a subalgebra of $\mathcal{A}$.

\end{lemma}

\begin{proof}
TODO?

\end{proof}

\begin{definition}
\label{def-subalgebra-generated}
Given $(X, \{\sigma_i\})$ is an algebra, if $S$ is a subset of $X$, then \textbf{the subalgebra generated by $S$} is the intersection of all subalgebras containing $S$.
\end{definition}

\begin{definition}
\label{def-algebra-induced-from}
Given $(X, \{\sigma_i\})$ is an algebra, if $A$ is any set, then \textbf{the algebra induced from $A$} is the algebra $(X^A, \{\box_i\})$, where we define for $f_1, \ldots, f_k \in X^A$ and each $\box_i$, the function $\box_i(f_1, \ldots, f_k)(a) := \sigma_i(f_1(a), \ldots, f_k(a))$ for $\sigma_i$ with $ar(\sigma_i) = k$. This leads to a well-defined algebra on $X^A$ that is similar to the original algebra.
\end{definition}

\end{document}
